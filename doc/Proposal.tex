\documentclass[a4paper, 12pt]{article}
 
\usepackage{indentfirst}
\usepackage[utf8]{inputenc} 
\usepackage[linesnumbered,boxed]{algorithm2e}
\usepackage[pdftex]{graphicx} 
\usepackage{amsmath}
\usepackage{amsfonts}
\usepackage[left=2cm,right=1.5cm,top=1.5cm,bottom=1.5cm]{geometry}
 
 
\newcommand{\mtrx}[1]{\ensuremath{\begin{pmatrix}#1\end{pmatrix}}}
\let\eps\varepsilon



\author{Dmytro Fedoriaka, Oleg Gorodnitskii}

\begin{document}


\begin{center}

{\Large Project proposal}
 
 
\end{center}

\bigskip
 


\section{Project name.} 

Optimal portfolio selection.

\section{Team.}

\begin{itemize}
  \item{\textbf{Oleg Gorodnitskii}}
	\item{\textbf{Dmytro Fedoriaka}}
	\item{\textbf{Rasul Khasianov}}
\end{itemize}

\section{Background.}

Investing is risky business. It's hard to predict how prices on market will change. Investing in several sequrities at once
can decrease risk. We can apply theory of probability and optimization methods to find the optimal strategy, which will
not only maximize expectation of profit, but also give guarantees about risk.

\section{Problem formulation.}

There is an investor. He possesses some initial wealth.
He can allocate it on the market for one period. There are n securities (shares, currencies, etc.) available on the market. He wants to buy some of them to maximize expectation of his wealth after that period. 
Problem is to select \textbf{optimal portfolio} – decide, how many of each security to buy.

This problem consists of two parts:
\begin{itemize}
\item{Model (predict) changes of prices of securities, using information about them in past.}
\item{Using model from (a), formulate and solve optimization problem.}
\end{itemize}


\section{Data.}

We will use historical data about quotations, extracted from \cite{finam}.

\section{Related work.}

\textbf{Markovits theory.} We want to start with Markowitz theory of portfolio selection \cite{mark}.

Let’s assume that for each paper we know profitability $r_i$ and we know matrix of correlation between profitabilities $\Sigma$.

Let's denote $x_i$ --- ratio of capital invested in security $i$. Then profit is $r^Tx$ and risk is $r^T \Sigma x$.
We can formulate problem as minimizing risk while maintaining peofit not less than given constant $p$:

$$\begin{cases}
  x^T \Sigma x \to \underset{x}{\min} \\
	r^T x  \ge p \\
	1^T x \le 1
\end{cases}$$

Or, we can formulate problem as maximizing profit while maintaining risk not too big:

$$\begin{cases}
  r^T x \to \underset{x}{\max} \\
	x^T \Sigma x \le \sigma\\
	1^T x \le 1
\end{cases}$$

Or, we can introduce regularizing paramater $\lambda$:

$$\begin{cases} 
	r^T x \Sigma x - \lambda x^T \Sigma x \to \underset{x}{\max} \\
	1^T x \le 1
\end{cases}$$

We can see that those are LP problems. There can be some additional requirements (like we can buy only integer amount of stock).
We plan to start with this model and than try to improve it. 


\textbf{Prediction.} Very important is to estimate profitability of securities (which is equivalent to estimating price in 
the end of period). For that we need to use time series prediction. We plan to start with GARCH \cite{engle}.


\section{Scope.}

The end result of the project will be an algorithm which takes as input historical data about 
quotations of sequrities, initial capital and duration of investment period and outputs the optimal portfolio.


\section{Evalution.}

We plan to check our solution on real data: select some moment in time, build optimal portfolio. Then pretend that we have
purchased the sequrities according to algorithm output and count resulting profit.


\section{References}
\def\refname{}
\begin{thebibliography}
{1}\bibitem{mark} Markowitz, Harry. "Portfolio selection." The journal of finance 7.1 (1952): 77-91.


\bibitem{engle}Engle, Robert F. "Autoregressive conditional heteroscedasticity with estimates of the variance of United Kingdom inflation." Econometrica: Journal of the Econometric Society (1982): 987-1007.

\bibitem{finam} {https://www.finam.ru}
\end{thebibliography}
\end{document}

